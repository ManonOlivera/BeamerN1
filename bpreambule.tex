\usetheme{Malmoe}
\usepackage[utf8]{inputenc}
\usepackage[french]{babel}
\usepackage[T1]{fontenc}
\usepackage{amsmath}
\usepackage{amsfonts}
\usepackage{amssymb}
\usepackage{graphicx}
\usepackage{hyperref}
\author{Manon OLIVERA}
%\usecolortheme[named={white}]{structure}
\setbeamercolor{structure}{fg=blue!40!white,bg=white}
%\setbeamercovered{transparent} 
%\setbeamertemplate{navigation symbols}{} 
%\logo{} 
%\institute{} 
%\date{} 
%\subject{}

\usepackage[skins,theorems,breakable]{tcolorbox}
\usepackage{fontawesome5}
\usepackage{siunitx}
\usepackage{pstricks-add}
\usepackage{tabularx}
\usepackage{enumitem}
\usepackage{hyperref}
\usepackage{xcolor}
\usepackage{varwidth}

\setbeamertemplate{navigation symbols}{%
%\insertslidenavigationsymbol
%\insertframenavigationsymbol
%\insertsubsectionnavigationsymbol
%\insertsectionnavigationsymbol
%\insertdocnavigationsymbol
%\insertbackfindforwardnavigationsymbol
}

\newtcbtheorem{defi}{Définition\,}{
theorem style = plain, 
%oversize,
enhanced, 
breakable,
before skip = 2mm, 
after skip = 2mm, 
top = 1mm,
left = 1mm,
right = 1mm,
bottom = 1mm,
coltitle=black!50!blue,
colback=blue!5, 
colframe=blue!5,
terminator sign dash, 
fonttitle=\bfseries}{df}

\newtcbtheorem{exemple}{Exemple\,}{
theorem style = plain, 
%oversize,
enhanced, 
breakable,
before skip = 2mm, 
after skip = 2mm,
top = 1mm,
left = 1mm,
right = 1mm,
bottom = 1mm, 
coltitle=black,
colback=white, 
colframe=white,
terminator sign dash, 
fonttitle=\bfseries}{ex}

\newtcbtheorem{remarque}{Remarque\,}{
theorem style = plain, 
%oversize,
enhanced, 
breakable,
before skip = 2mm, 
after skip = 2mm,
top = 1mm,
left = 1mm,
right = 1mm,
bottom = 1mm, 
coltitle=black,
colback=white, 
colframe=white,
terminator sign dash, 
fonttitle=\bfseries}{rq}

\newtcbtheorem{prop}{Propriété\,}{
theorem style = plain, 
%oversize,
enhanced, 
breakable,
before skip = 2mm, 
after skip = 2mm, 
top = 1mm,
left = 1mm,
right = 1mm,
bottom = 1mm,
coltitle=black!50!orange,
colback=orange!5, 
colframe=orange!5,
terminator sign dash, 
fonttitle=\bfseries}{ppt}	

\newtcbtheorem{notation}{Notation\,}{
theorem style = plain, 
%oversize,
enhanced, 
breakable,
before skip = 2mm, 
after skip = 2mm,
top = 1mm,
left = 1mm,
right = 1mm,
bottom = 1mm, 
coltitle=black,
colback=white, 
colframe=white,
terminator sign dash, 
fonttitle=\bfseries}{not}

\newtcbtheorem{exercice}{Exercice}{
enhanced, 
breakable,
before skip = 2mm, 
after skip = 2mm,
top = 1mm,
left = 1mm,
right = 1mm,
bottom = 1mm, 
coltitle=white,
colbacktitle=black,
colback=white, 
colframe=black,
boxrule=0.3mm,
terminator sign none, 
fonttitle=\bfseries,
attach boxed title to top left={xshift=1cm,yshift*=1mm-\tcboxedtitleheight},
varwidth boxed title*=-3cm,
boxed title style={frame code={
            \path[fill=tcbcolback!30!black]
              ([yshift=-1mm,xshift=-1mm]frame.north west)
                arc[start angle=0,end angle=180,radius=1mm]
              ([yshift=-1mm,xshift=1mm]frame.north east)
                arc[start angle=180,end angle=0,radius=1mm];
            \path[left color=tcbcolback!60!black,right color=tcbcolback!60!black,
              middle color=tcbcolback!80!black]
              ([xshift=-2mm]frame.north west) -- ([xshift=2mm]frame.north east)
              [rounded corners=1mm]-- ([xshift=1mm,yshift=-1mm]frame.north east)
              -- (frame.south east) -- (frame.south west)
              -- ([xshift=-1mm,yshift=-1mm]frame.north west)
              [sharp corners]-- cycle;
            },interior engine=empty,
          }}{exo}