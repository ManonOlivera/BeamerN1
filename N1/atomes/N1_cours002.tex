\begin{defi*}{}{}
Les nombres sont regroupés en \textbf{classes} composées de trois rangs : unités, dizaines, centaines. On peut représenter ces données dans un tableau.

\renewcommand{\arraystretch}{1.8}
\begin{center}
    \scalebox{0.8}{
    \begin{tabular}{*{3}{|c}|*{3}{|c}|*{3}{|c}|*{3}{|c}|}
        \hline
        \multicolumn{3}{|c||}{\textbf{Classe des Milliards}} & \multicolumn{3}{c||}{\textbf{Classe des Millions}}&\multicolumn{3}{c||}{\textbf{Classe des Milliers}}&\multicolumn{3}{c|}{\textbf{Classe des Unités}}\\
        \hline
        \rotatebox{90}{\parbox{3.5cm}{\textbf{Centaines}\\ de milliards }}& 
        \rotatebox{90}{\parbox{3.5cm}{\textbf{Dizaines}\\ de milliards }}& 
        \rotatebox{90}{\parbox{3.5cm}{\textbf{Unités}\\ de milliards }}& 
        \rotatebox{90}{\parbox{3.5cm}{\textbf{Centaines}\\ de millions }}& 
        \rotatebox{90}{\parbox{3.5cm}{\textbf{Dizaines}\\ de millions }}& 
        \rotatebox{90}{\parbox{3.5cm}{\textbf{Unités}\\ de millions }}& 
        \rotatebox{90}{\parbox{3.5cm}{\textbf{Centaines}\\ de milliers }}& 
        \rotatebox{90}{\parbox{3.5cm}{\textbf{Dizaines}\\ de milliers }}& 
        \rotatebox{90}{\parbox{3.5cm}{\textbf{Unités}\\ de milliers }}& 
        \rotatebox{90}{\parbox{3.5cm}{\textbf{Centaines}\\ d'unités }}& 
        \rotatebox{90}{\parbox{3.5cm}{\textbf{Dizaines}\\ d'unités }}&
        \rotatebox{90}{\parbox{3.5cm}{\textbf{Unités}\\ d'unités}} 
        \\
        \hline
        & & & & 1 & 2 & 0 & 4 & 5 & 9 & 7 & 6 \\
        \hline
    \end{tabular} 
    }
   \end{center}
\end{defi*}
   
\begin{exemple*}{}{}
Dans le nombre \num{12045976} :
\begin{itemize}[label=$-$]
\begin{minipage}[t]{0.55\linewidth}
	\item 9 est le chiffre des centaines ;
	\item 5 est le chiffre des milliers ;
	\item 1 est le chiffre des dizaines de millions ;
\end{minipage}
\begin{minipage}[t]{0.38\linewidth}
	\item il y a \num{120459} centaines ;
	\item il y a \num{1204} dizaines de milliers ;
	\item il y a 12 millions.
\end{minipage}
\end{itemize}
\end{exemple*}