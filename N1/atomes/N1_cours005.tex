\begin{prop*}{Règles orthographiques}{}
\begin{itemize}[label=$-$]
	\item Deux mots d'un même nombre sont séparés par un trait d'union.
	\item \og Mille \fg \, est invariable.
	\item \og Cent \fg \, ou \og vingt \fg \, prennent la marque du pluriel, \og s \fg, sauf quand ils sont suivis d'un autre adjectif numéral (\og quatre \fg{} par exemple). Toutefois, devant \og millier \fg, \og million \fg \, ou \og milliard \fg, qui sont des noms, le \og s \fg \, du pluriel subsiste.
	\item \og Million \fg \, et \og milliard \fg \, prennent toujours un \og s \fg{} quand ils sont au pluriel.
\end{itemize}
\end{prop*}

\begin{exemple*}{}{}
\begin{itemize}[label=$-$]
	\item \num{4000} : quatre-mille ;        
	\item \num{12045976} : douze-millions-quarante-cinq-mille-neuf-cent-soixante-seize.
	\item $80$ s'écrit "quatre-vingt\underline{s}" mais $83$ s'écrit "quatre-vingt-trois"
	\item $200$ s'écrit "deux-cent\underline{s}" mais $237$ s'écrit "deux-cent trente-sept"
	\item Deux-cents personnes sont attendues, mais établissez un chèque de cinq-cent quarante euros
\end{itemize}
\end{exemple*}