\documentclass[10pt,a4paper]{report}
\usepackage[utf8]{inputenc}
\usepackage[french]{babel}
\usepackage[T1]{fontenc}
\usepackage{amsmath}
\usepackage{amsfonts}
\usepackage{amssymb}
\usepackage{lmodern}
\usepackage[left=2cm,right=2cm,top=2cm,bottom=2cm]{geometry}
\usepackage[skins,theorems,breakable]{tcolorbox}
\usepackage{siunitx}
\usepackage{pstricks-add}
\usepackage{pst-eucl}
\usepackage{tabularx}
\usepackage{enumitem}
\usepackage{hyperref}
\usepackage{varwidth}

\renewcommand{\thesection}{\arabic{section}.}
\setlength{\parindent}{0pt}

\newtcbtheorem{defi}{Définition\,}{
theorem style = plain, 
%oversize,
enhanced, 
breakable,
before skip = 2mm, 
after skip = 2mm, 
top = 2mm,
left = 2mm,
right = 2mm,
bottom = 2mm,
coltitle=black!50!blue,
colback=blue!5, 
colframe=blue!5,
terminator sign dash, 
fonttitle=\bfseries}{df}

\newtcbtheorem{exemple}{Exemple\,}{
theorem style = plain, 
%oversize,
enhanced, 
breakable,
before skip = 2mm, 
after skip = 2mm,
top = 2mm,
left = 2mm,
right = 2mm,
bottom = 2mm, 
coltitle=black,
colback=white, 
colframe=white,
terminator sign dash, 
fonttitle=\bfseries}{ex}

\newtcbtheorem{remarque}{Remarque\,}{
theorem style = plain, 
%oversize,
enhanced, 
breakable,
before skip = 2mm, 
after skip = 2mm,
top = 2mm,
left = 2mm,
right = 2mm,
bottom = 2mm,
coltitle=black,
colback=white, 
colframe=white,
terminator sign dash, 
fonttitle=\bfseries}{rq}

\newtcbtheorem{prop}{Propriété\,}{
theorem style = plain, 
%oversize,
enhanced, 
breakable,
before skip = 2mm, 
after skip = 2mm, 
top = 2mm,
left = 2mm,
right = 2mm,
bottom = 2mm,
coltitle=black!50!orange,
colback=orange!5, 
colframe=orange!5,
terminator sign dash, 
fonttitle=\bfseries}{ppt}	

\newtcbtheorem{notation}{Notation\,}{
theorem style = plain, 
%oversize,
enhanced, 
breakable,
before skip = 2mm, 
after skip = 2mm,
top = 2mm,
left = 2mm,
right = 2mm,
bottom = 2mm, 
coltitle=black,
colback=white, 
colframe=white,
terminator sign dash, 
fonttitle=\bfseries}{not}

\newtcbtheorem{exercice}{Exercice}{
enhanced, 
%breakable,
before skip = 2mm, 
after skip = 2mm,
top = 2mm,
left = 2mm,
right = 2mm,
bottom = 2mm, 
coltitle=white,
colbacktitle=gray,
colback=white, 
colframe=gray,
boxrule=0.3mm,
terminator sign none, 
fonttitle=\bfseries,
attach boxed title to top left={xshift=1cm,yshift*=1mm-\tcboxedtitleheight},
varwidth boxed title*=-3cm,
boxed title style={frame code={
            \path[fill=tcbcolback!30!black]
              ([yshift=-1mm,xshift=-1mm]frame.north west)
                arc[start angle=0,end angle=180,radius=1mm]
              ([yshift=-1mm,xshift=1mm]frame.north east)
                arc[start angle=180,end angle=0,radius=1mm];
            \path[left color=tcbcolback!60!black,right color=tcbcolback!60!black,
              middle color=tcbcolback!80!black]
              ([xshift=-2mm]frame.north west) -- ([xshift=2mm]frame.north east)
              [rounded corners=1mm]-- ([xshift=1mm,yshift=-1mm]frame.north east)
              -- (frame.south east) -- (frame.south west)
              -- ([xshift=-1mm,yshift=-1mm]frame.north west)
              [sharp corners]-- cycle;
            },interior engine=empty,
          }}{exo}    

\begin{document}

\hrulefill
\vspace{3mm}
\begin{center}
{\Huge \textbf{Chapitre N1 - Nombres entiers}}
\end{center}
\hrulefill

\section{Écrire des nombres entiers}

\begin{defi*}{}{}
Dans notre numération, il y a dix \textbf{chiffres} : 0, 1, 2, 3, 4, 5, 6, 7, 8 et 9. Ces chiffres permettent d'écrire des \textbf{nombres entiers} : il y en a une infinité.
\end{defi*}

\begin{exemple*}{}{}
\num{1054} est un nombre composé de quatre chiffres. 7 est un nombre composé d'un seul chiffre.
\end{exemple*}
\begin{defi*}{}{}
Les nombres sont regroupés en \textbf{classes} composées de trois rangs : unités, dizaines, centaines. On peut représenter ces données dans un tableau.

\renewcommand{\arraystretch}{1.8}
\begin{center}
    \scalebox{0.8}{
    \begin{tabular}{*{3}{|c}|*{3}{|c}|*{3}{|c}|*{3}{|c}|}
        \hline
        \multicolumn{3}{|c||}{\textbf{Classe des Milliards}} & \multicolumn{3}{c||}{\textbf{Classe des Millions}}&\multicolumn{3}{c||}{\textbf{Classe des Milliers}}&\multicolumn{3}{c|}{\textbf{Classe des Unités}}\\
        \hline
        \rotatebox{90}{\parbox{3.5cm}{\textbf{Centaines}\\ de milliards }}& 
        \rotatebox{90}{\parbox{3.5cm}{\textbf{Dizaines}\\ de milliards }}& 
        \rotatebox{90}{\parbox{3.5cm}{\textbf{Unités}\\ de milliards }}& 
        \rotatebox{90}{\parbox{3.5cm}{\textbf{Centaines}\\ de millions }}& 
        \rotatebox{90}{\parbox{3.5cm}{\textbf{Dizaines}\\ de millions }}& 
        \rotatebox{90}{\parbox{3.5cm}{\textbf{Unités}\\ de millions }}& 
        \rotatebox{90}{\parbox{3.5cm}{\textbf{Centaines}\\ de milliers }}& 
        \rotatebox{90}{\parbox{3.5cm}{\textbf{Dizaines}\\ de milliers }}& 
        \rotatebox{90}{\parbox{3.5cm}{\textbf{Unités}\\ de milliers }}& 
        \rotatebox{90}{\parbox{3.5cm}{\textbf{Centaines}\\ d'unités }}& 
        \rotatebox{90}{\parbox{3.5cm}{\textbf{Dizaines}\\ d'unités }}&
        \rotatebox{90}{\parbox{3.5cm}{\textbf{Unités}\\ d'unités}} 
        \\
        \hline
        & & & & 1 & 2 & 0 & 4 & 5 & 9 & 7 & 6 \\
        \hline
    \end{tabular} 
    }
   \end{center}
\end{defi*}
   
\begin{exemple*}{}{}
Dans le nombre \num{12045976} :
\begin{itemize}[label=$-$]
\begin{minipage}[t]{0.55\linewidth}
	\item 9 est le chiffre des centaines ;
	\item 5 est le chiffre des milliers ;
	\item 1 est le chiffre des dizaines de millions ;
\end{minipage}
\begin{minipage}[t]{0.38\linewidth}
	\item il y a \num{120459} centaines ;
	\item il y a \num{1204} dizaines de milliers ;
	\item il y a 12 millions.
\end{minipage}
\end{itemize}
\end{exemple*}
\begin{defi*}{}{}
On peut décomposer tout nombre sous sa \textbf{forme canonique} :
\begin{align*}
\num{12045976} = \; &\num{10000000} + \num{2000000} + \num{40000} + \num{5000} + 900 + 70 + 2 \\
= \; &(1\times\num{10000000}) + (2\times\num{1000000}) + (4\times\num{10000}) + (5\times\num{1000}) \\
\; &+ (9\times100) +(7\times10) + ( 6\times1)
\end{align*}
\end{defi*}
\vspace{2mm}
\begin{prop*}{}{}
Pour pouvoir lire les grands nombres plus facilement, on regroupe les chiffres par tranches de trois en partant du chiffre des unités de la classe des unités.
\end{prop*}

\begin{exemple*}{}{}
\begin{itemize}[label=$-$]
	\item $12345678910111213$ s'écrira plutôt $\num{12345678910111213}$.
	\item $9123456789$ s'écrira plutôt $\num{9123456789}$, et se lit \og neuf \underline{milliards} cent vingt-trois \underline{millions} quatre cent cinquante-six \underline{mille} sept cent quatre-vingt-neuf \underline{unités} \fg
\end{itemize}
\end{exemple*}
\begin{prop*}{Règles orthographiques}{}
\begin{itemize}[label=$-$]
	\item Deux mots d'un même nombre sont séparés par un trait d'union.
	\item \og Mille \fg \, est invariable.
	\item \og Cent \fg \, ou \og vingt \fg \, prennent la marque du pluriel, \og s \fg, sauf quand ils sont suivis d'un autre adjectif numéral (\og quatre \fg{} par exemple). Toutefois, devant \og millier \fg, \og million \fg \, ou \og milliard \fg, qui sont des noms, le \og s \fg \, du pluriel subsiste.
	\item \og Million \fg \, et \og milliard \fg \, prennent toujours un \og s \fg{} quand ils sont au pluriel.
\end{itemize}
\end{prop*}

\begin{exemple*}{}{}
\begin{itemize}[label=$-$]
	\item \num{4000} : quatre-mille ;        
	\item \num{12045976} : douze-millions-quarante-cinq-mille-neuf-cent-soixante-seize.
	\item $80$ s'écrit "quatre-vingt\underline{s}" mais $83$ s'écrit "quatre-vingt-trois"
	\item $200$ s'écrit "deux-cent\underline{s}" mais $237$ s'écrit "deux-cent trente-sept"
	\item Deux-cents personnes sont attendues, mais établissez un chèque de cinq-cent quarante euros
\end{itemize}
\end{exemple*}

\section{La règle graduée}

\begin{defi*}{}{}
Pour graduer une droite, il faut choisir :
\begin{itemize}[label=$-$]
	\item une {\bf origine} qui correspond au \og 0 \fg{},
	\item une {\bf unité} qui sera reportée de manière régulière,
	\item un {\bf sens croissant}.
\end{itemize}
Un point est repéré par son {\bf abscisse}. A a pour abscisse 3 se note A(3). \\
\begin{pspicture}(-2,-1)(5.5,1)
   \psset{xunit=2}
      \psaxes[yAxis=false]{->}(0,0)(5.2,1)
      \psline[linecolor=violet]{<-}(-0.02,0.04)(-0.3,0.5)
      \rput(-0.6,0.5){\textcolor{violet}{origine}}
      \psline[linecolor=blue]{<->}(0,0.3)(1,0.3)
      \rput(0.5,0.6){\textcolor{blue}{unité}}
      \rput(3,0.4){\textcolor{red}{A}}
      \psline[linecolor=red]{<-}(3.02,0.04)(3.3,0.5)
      \rput(4.1,0.5){\textcolor{red}{l'abscisse de A est 3}}
      \psline[linecolor=orange]{<-}(5.2,0)(5.5,0.5)
      \rput(5.9,0.8){\textcolor{orange}{sens croissant}}
   \end{pspicture}   
\end{defi*}

\begin{exemple*}{}{}
\begin{center}
\begin{pspicture}(0,-0.5)(10.5,0.5)
      \psaxes[yAxis=false,labels=none]{->}(0,0)(10.5,0)
      \rput(0,-0.4){0}
      \rput(1,-0.4){100}
      \rput(3,0.4){A}
      \rput(7,0.4){R}
      \rput(10,0.4){C}
   \end{pspicture}
\end{center}
Ici, l'unité vaut 100, donc les points A, R et C ont pour abscisses 300; 700 et \num{1000}. On note  A(300), R(700) et C(\num{1000}).
\end{exemple*}

\section{Ordonner des nombres entiers}

\begin{defi*}{}{}
{\bf Comparer} deux nombres, c'est dire s'ils sont égaux ou si l'un est plus petit (ou plus grand) que l'autre.
\end{defi*}

\begin{notation*}{}{}
Dans notre sens de lecture (de gauche à droite), le symbole \fbox{<} signifie \og plus petit que \fg{} et \fbox{>} signifie \og plus grand que \fg{}.  
\end{notation*}

\begin{exemple*}{}{}
\begin{itemize}[label=$-$]
	\item $\num{1000000200} > \num{1000000002}$ se lit \og $\num{1000000200}$ est plus grand que $\num{1000000002}$ \fg{}. 
	\item $\num{999999}<\num{1000000}$ si lit \og $\num{999999}$ est plus petit que $\num{1000000}$ \fg{}.
\end{itemize}
\end{exemple*}
\begin{defi*}{}{}
\begin{itemize}[label=$-$]
	\item Ranger des nombres dans l'ordre {\bf croissant} signifie les ranger du plus petit au plus grand.
	\item Ranger des nombres dans l'ordre {\bf décroissant} signifie les ranger du plus grand au plus petit.
\end{itemize}
\end{defi*}

\begin{exemple*}{}{}
\begin{itemize}[label=$-$]
	\item $\num{1000045}<\num{1000085}<\num{1000600}<\num{1000607}$ sont rangés dans l'ordre croissant.
	\item $321>312>231>213>132>123$ sont rangés dans l'ordre décroissant.
	\end{itemize}
\end{exemple*}
\begin{defi*}{}{}
{\bf Encadrer} un nombre, c'est l'entourer par un nombre plus petit et un nombre plus grand.
\end{defi*}

\begin{exemple*}{}{}
On peut encadrer le nombre 8\,199 de différentes façons, par exemple :   
\begin{itemize}
	\item $\num{8198}<\num{8199}<\num{8200}$
	\item $\num{8000}<\num{8199}<\num{9000}$
	\item $\num{1000}<\num{8199}<\num{10000}$\dots
\end{itemize}
\end{exemple*}

\newpage

\setcounter{section}{0}

\hrulefill
\vspace{3mm}
\begin{center}
{\Huge \textbf{Chapitre N1 - Exercices}}
\end{center}
\hrulefill

\section{Écrire des nombres entiers}

\begin{exercice}{}{}
Écrire en chiffre les nombres suivants :
\begin{enumerate}
	\item Sept-milliards-cinq-cent-cinquante-neuf-millions-deux-cent-quatre-vingt-huit-mille-trois-cents.
	\item Neuf-millions-sept-cent-mille-sept-cent-quarante. 
   \item Trente-huit-millions-trente-huit-mille.
   \item Vingt-six-milliards-cent-huit-millions-sept-cent-vingt-huit-mille-douze.
\end{enumerate}
\end{exercice}
\begin{exercice}{}{}
    Voici cinq cartes contenant un nombre :
    \begin{center}
       \fbox{415} \qquad \fbox{2\,103} \qquad \fbox{9} \qquad \fbox{87} \qquad\fbox{13}
    \end{center}
    Placer ces cartes côte à côte pour écrire :
    \begin{enumerate}
       \item le plus petit nombre entier de douze chiffres ;
       \item le plus grand nombre entier.
    \end{enumerate}
 \end{exercice}
\begin{exercice}{}{}
    Dans le nombre $\num{6083472}$ donner :
    \begin{enumerate}
       \item le chiffre des unités ;
       \item le chiffre des dizaines de mille ;
       \item le chiffre des unités de millions ;
       \item le nombre de centaines ;
       \item le nombre de centaines de mille ;
       \item le nombre de millions.
    \end{enumerate}
 \end{exercice}
\begin{exercice}{}{}
    Écrire en chiffres chacun des nombres.
    \begin{enumerate}
       \item 13 centaines et 25 unités.
       \item 43 millions et 8 dizaines.
       \item 25 dizaines de mille et 67 centaines.
       \item 12 dizaines de milliards et 3 centaines de millions.
    \end{enumerate}
 \end{exercice}
\begin{exercice}{}{}
    Écrire le résultat des opérations :
     \begin{enumerate}
        \item $(1\times1\,000) + (4\times100) + (8\times10)$
        \item $(3\times100\,000) + (6\times10\,000) + (1\times10)$
        \item $(2\times1\,000\,000) + (9\times1\,000) + (5\times1)$
        \item $(7\times1\,000\,000\,000) + (7\times1\,000) + (3\times100)$
     \end{enumerate}
\end{exercice}
\begin{exercice}{}{}

\end{exercice}
\begin{exercice}{}{}
    Écrire en lettres les nombres suivants :
    \begin{enumerate}
       \item $\num{999}$.
       \item $\num{58736}$.
       \item $\num{53200000}$.
       \item $\num{543823942900}$.
    \end{enumerate}
\end{exercice}

\section{La règle graduée}

\begin{exercice}{}{}
    Écrire l'abscisse de chacun des points représentés sur la droite graduée.
    \begin{enumerate}
    \small
       \item \begin{pspicture}(0,0)(8,0.7)
                   \psset{xunit=5}
                   \psaxes[yAxis=false,subticks=10,subtickcolor=gray]{->}(0,0)(1.5,0)
                   \pstGeonode[PosAngle=90](0.3,0){A}(0.8,0){B}(1.1,0){C}(1.3,0){D}
                   \rput(1.024,-0.415){0}
                \end{pspicture}
       \item \begin{pspicture}(0,0)(8,1.2)
                   \psset{xunit=1.15}
                   \psaxes[yAxis=false,subticks=2,subtickcolor=gray,labels=none]{->}(0,0)(6.5,0)
                   \rput(1,-0.4){1\,000}
                   \rput(2,-0.4){2\,000}
                   \pstGeonode[PosAngle=90](1.5,0){J}(3.5,0){K}(5.5,0){L}
                \end{pspicture}
       \item \begin{pspicture}(0,0)(8,1.2)
                   \psset{xunit=0.47}
                   \psline{->}(0,0)(16,0)
                   \multido{\n=0+1}{15}{\psline[linecolor=gray,linewidth=0.1mm](\n,-0.1)(\n,0.1)}
                   \psline(3,-0.1)(3,0.1)
                   \rput(3,-0.4){29\,000}
                   \psline(13,-0.1)(13,0.1)
                   \rput(13,-0.4){30\,000}
                   \pstGeonode[PosAngle=90](2,0){R}(7,0){S}(10,0){T}(14,0){U}
                \end{pspicture} \\
    \end{enumerate}
\end{exercice}
\begin{exercice}{}{}
    Placer les points dont l'abscisse est donnée sur les droites graduées. \smallskip
    \begin{enumerate}
    \small
       \item E(121) \qquad F(123) \qquad G(125) \qquad H(131) \\
          \begin{pspicture}(0,-0.8)(8,0.7)
                   \psset{xunit=5}
                   \psaxes[yAxis=false,Ox=120,dx=1,Dx=10,subticks=10,subtickcolor=gray]{->}(0.2,0)(1.5,0)
                   \psline(0,0)(0.2,0)
                   \psline[linewidth=0.07mm](0.1,0.1)(0.1,-0.1)
                \end{pspicture} 
       \item M(7\,810) \qquad N(7\,830) \qquad P(7\,890) \qquad Q(7\,910) \\
          \begin{pspicture}(0,-0.8)(8,0.7)
                   \psset{xunit=0.45}
                   \psline{->}(0,0)(16,0)
                   \multido{\n=0+1}{16}{\psline[linecolor=gray,linewidth=0.1mm](\n,-0.1)(\n,0.1)}
                   \psline(2,-0.1)(2,0.1)
                   \rput(2,-0.4){7\,800}
                   \psline(12,-0.1)(12,0.1)
                   \rput(12,-0.4){7\,900}
                \end{pspicture}
        \item V(640\,800) \qquad W(641\,300) \qquad Y(641\,600) \qquad Z(641\,800) \\
           \begin{pspicture}(0,-0.5)(8,0.7)
                   \psset{xunit=0.45}
                   \psline{->}(0,0)(16,0)
                   \multido{\n=0+1}{16}{\psline[linecolor=gray,linewidth=0.1mm](\n,-0.1)(\n,0.1)}
                   \psline(4,-0.1)(4,0.1)
                   \rput(4,-0.4){641\,000}
                   \psline(14,-0.1)(14,0.1)
                   \rput(14,-0.4){642\,000}
                \end{pspicture}
    \end{enumerate}
\end{exercice}

\section{Ordonner des nombres entiers}

\begin{exercice}{}{}

\end{exercice}
\begin{exercice}{}{}
    Ranger chaque série de nombres :
    \begin{enumerate}
       \item dans l'ordre croissant. \\ [1mm]
          \fbox{$\num{1 110}$} \; \fbox{$\num{1 101}$} \; \fbox{$\num{1 011}$} \; \fbox{$\num{1 111}$} \; \fbox{$\num{1 100}$} \; \fbox{$\num{1 010}$} \smallskip
       \item dans l'ordre décroissant. \\ [1mm]
          \fbox{$128$} \; \fbox{$182$} \; \fbox{$281$} \; \fbox{$218$} \; \fbox{$280$} \; \fbox{$821$} \; \fbox{$812$}
    \end{enumerate}
\end{exercice}
\begin{exercice}{}{}
    Encadrer avec l'entier précédent et suivant. \medskip
    \begin{enumerate}
       \item \makebox[0.3\linewidth]{\dotfill} < \makebox[0.2\linewidth]{$\num{850}$   } <   \makebox[0.3\linewidth]{\dotfill}\mbox{} \medskip
       \item \makebox[0.3\linewidth]{\dotfill} < \makebox[0.2\linewidth]{$\num{9 901}$ } <   \makebox[0.3\linewidth]{\dotfill}\mbox{} \medskip
       \item \makebox[0.3\linewidth]{\dotfill} < \makebox[0.2\linewidth]{$\num{956}$   } <   \makebox[0.3\linewidth]{\dotfill}\mbox{} \medskip
       \item \makebox[0.3\linewidth]{\dotfill} < \makebox[0.2\linewidth]{$\num{29 008}$} <   \makebox[0.3\linewidth]{\dotfill}\mbox{} \medskip
       \item \makebox[0.3\linewidth]{\dotfill} < \makebox[0.2\linewidth]{$\num{12 309}$} <   \makebox[0.3\linewidth]{\dotfill}\mbox{} \medskip
       \item \makebox[0.3\linewidth]{\dotfill} < \makebox[0.2\linewidth]{$\num{77 777}$} <   \makebox[0.3\linewidth]{\dotfill}\mbox{}
    \end{enumerate}
\end{exercice}

\end{document}